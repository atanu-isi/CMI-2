\documentclass[12pt,a4paper,oneside]{amsart}
%Title details
\usepackage{hyperref, fancyhdr}
\oddsidemargin -0.3in
\evensidemargin -0.3in
\textwidth 7.0in
\textheight 9.5in
\parindent=0.5in
\pagestyle{fancy}
\title{Statement of Purpose}
\author{\small{Arnab Kar, Date of Birth November 3, 1987, Applying for Ph.D. in Physics, Fall 2009}}
\begin{document}

\maketitle
\fancyhead{}
\fancyfoot{}

\rhead{\small{Arnab Kar, Date of Birth November 3, 1987, Applying for Ph.D. in Physics, Fall 2009}}
\cfoot{\thepage}

I am applying for the admission to the Doctoral programme in Physics. In general, my interests in Physics span various fields, but I am primarily interested in Elementary Particle Physics. I desire to make significant contributions to my field, and also involve myself in teaching.

The techniques of logical deductions, approximations and assumptions used in physics to explain various phenomena have always delighted me. During my high school days I used to be a regular reader of the magazine Physics for you. I have also spent a lot of time solving problems from \textit{I.E. Irodov's} Problems in General Physics. I used to get a great deal of satisfaction after solving problems from that book. After High School I enrolled in the B.Sc.(Hons) programme at \textit{Chennai Mathematical Institute (CMI)}. CMI's National Undergraduate Programme's aim is to train a small, selected group of promising students by active researchers in their respective fields, so that they can be exposed to almost all major frontiers of research. The faculty compromises of professors from the \textit{Institute of Mathematical Sciences (IMSc), Indira Gandhi Center for Atomic Research (IGCAR)} and other premier institutes of India in addition to CMI itself.

A quality undergraduate programme in Physics, from Chennai Mathematical Institute (CMI) has helped me acquire a strong foothold in physics along with a sound understanding and intuition. I have benefited from the well rounded curriculum \footnote{Course Details:\url{http://www.cmi.ac.in/teaching/courses.php?prog=bscp}} at CMI. The immense power of mathematics as a tool to understand, model and predict physical phenomena has been emphasized to a great extent over here. Courses in group theory, linear algebra, calculus, complex analysis along with mathematical physics have helped me develop a strong foundation in mathematics. 

I covered classical physics in my first year including courses on Classical Mechanics, Thermodynamics and Statistical Mechanics. I did a great deal of problem solving in my fundamental courses and referred to texts like \textit{Herbert Goldstein's} Classical Mechanics, \textit{Percival's} Introduction to Dynamics, \textit{Callen's} Thermodynamics and an Introduction to Thermostatistics and \textit{R.K. Pathria's} Statistical Mechanics. In my Classical Electromagnetism courses taken by Prof.~T.~R.Govindrajan and Prof.~R.~Jagannathan,  I almost read all the chapters from \textit{David Griffith's} Intrduction to Electrodynamics. I have done a course in General Relativity in the last semester where I read the \textit{Hartle's} book on Introduction to Einstein's General Relativity. 

In my third semester, I did my first course in Quantum Mechanics. I studied Non Relativistic Quantum Mechanics extensively from \textit{David Griffith's} Introduction to Quantum Mechanics. To have a better understanding of the mathematics used in the Quantum Mechanics I read the book of \textit{Dirac's} Principles of Quantum Mechanics. Independently I have also read the first six chapters of Lectures on Quantum Theory: Mathematical and Structural Foundations by \textit{C.J.Isham} and intend to read more from it. At the same time I was lucky to attend a series of lectures on Continuous Groups for physicists given by Prof. Mukunda in IMSc. During this course I learnt about Representation theory of Lie Groups. I read about the various space time groups and their representations namely the Galilean group, Lorentz group and Poincare group. At a later point of time I was able to appreciate those topics much better when I did a course on Relativistic Quantum Mechanics.

In my second course on Quantum Mechanics I learnt Perturbation theory and Scattering theory. In the Atomic and Molecular Physics course under Prof.~R.~Parthasarathy I came across a lot of application of scattering theory while calculating the differential cross sections for various interactions. The fact that proton is not a point particle but has a structure associated with it was highlighted by relating the root mean square radius and the charge form factor. The transition probabilities of electrons in the hydrogen atom due to interactions with light were calculated. The way creation and annihilation operators associated the radiation field (an assembly of quantum harmonic oscillators) are used to calculate the lifetime of spontaneous emission astounded me. 

In the following semester Prof.~R.~Parthasarathy taught us Relativistic Quantum Mechanics. During this course, the Dirac equation and its relativistic covariance, various discrete transformations (charge conjugation, time reversal, parity) of the wave function, Dirac's Hole Theory to solve the problem of negative energy solutions were covered in odetails. In Dirac's theory of Scattering I learnt how the spin flip cross section goes to zero in extreme relativistic case. I read a few chapters of \textit{Bjorken and Drell's} book on Relativistic Quantum Mechanics during the course. Spurred by this I started reading \textit{Sakurai's} Advanced Quantum Mechanics under the guidance of Prof.~G.~Rajshekaran. I intend to read more on quantum field theory and have recently started reading \textit{Schweber's} Introduction to Relativistic Quantum Field theory. 

Given my interests, Washington University is a natural choice for me. The research work being carried out on problems in quantum field theory and application of quantum field theory to solve problems in statistical mechanics interest me a lot. After having read an article (Phys. Rev. Lett. 80, 5243 (1998)). by Prof. Bender on Non-Hermitian Hamiltonians I am eager to know more on  this new field of complex quantum mechanics. In this regard, I have read an article on the arXiv on Dissipative Mechanics using Complex Valued Hamiltonians. The use of a simple system such as a harmonic oscillator to show how interaction with the environment leads to energy loss which cannot be neglected amazed me. I am eager to know more on the applications of this formalism to information theory. 

It is said that teaching is an integrated part of research. In order to have a better grasp on a topic that I learn, I have always enjoyed teaching others. During my General Relativity course I presented the topic of Rotating Black holes to my class. In the talk I highlighted the important features of a Kerr black hole mainly its ergosphere and the Penrose process associated with it. As a part of student talks held in my college I spoke about Special Relativity drawing reference to Michelson Morley experiment. I also delivered a talk on Adiabatic theorem in Classical and Quantum Mechanics. The details of the presentations are available on this website \footnote{\url{http://www.cmi.ac.in/~arnabkar/talk/talk.html}}. Based on these experiences, and after having been appreciated by the audience as a good speaker I have gained confidence on my abilities as a teaching assistantship.

In the next semester, I will be doing a course on Nuclear and Particle physics. Keeping in mind my long term objective, I would like to pursue a PhD as it would enable me to delve deeper into the subject I am interested in. I believe, Washington University with its research facilities and illustrious faculty is the best choice for me. With my diverse academic background and zeal for research, I sincerely hope that I will be given an opportunity to pursue my education and research in your esteemed institution. Washington will be a bright start for my future endeavors in research and teaching. I am confident that my performance will be at the fullest of my potential, and will be of value to the University.

\end{document}
