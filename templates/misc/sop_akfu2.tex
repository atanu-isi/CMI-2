\documentclass[12pt,a4paper,oneside]{amsart}

%Title details
\usepackage{hyperref, fancyhdr}
\topmargin 0.35in
\oddsidemargin -0.3in
\evensidemargin -0.3in
\textwidth 7.0in
\textheight 9.5in
\parindent=0.5in
\pagestyle{fancy}
\title{Statement of Purpose}
\author{\small{Arnab Kar, Date of Birth November 3, 1987, Applying for Ph.D. in Physics, Fall 2009}}

\begin{document}

\maketitle
\fancyhead{}
\fancyfoot{}

\rhead{\small{Arnab Kar, Date of Birth November 3, 1987, Applying for Ph.D. in Physics, Fall 2009}}
\cfoot{\thepage}

I am applying for the admission to the Ph.D. programme in Physics. My interests in Physics span various fields, but I am primarily interested in Particle Physics. I desire to make significant contributions to my field, and involve myself in teaching.

During my high school days I used to be a regular reader of the magazine Physics for you and solved problems from I.E. Irodov's Problems in General Physics. After High School I enrolled in the B.Sc.(Hons) programme at Chennai Mathematical Institute (CMI). CMI's National Undergraduate Programme's aim is to train a small, selected group of promising students by active researchers in their respective fields, so that they can be exposed to almost all major frontiers of research. 

I have benefited from the well rounded curriculum\footnote{Course Details:\url{http://www.cmi.ac.in/teaching/courses.php?prog=bscp}} at CMI. The immense power of mathematics as a tool to understand, model and predict physical phenomena has been emphasized here. Courses in group theory, linear algebra, calculus, complex analysis along with mathematical physics have helped me develop a strong foundation in mathematics. I covered classical physics in my first year including courses on Classical Mechanics, Classical Electromagnetism, Thermodynamics and Statistical Mechanics. I did a great deal of problem and referred to texts like Herbert Goldstein's Classical Mechanics, Percival's Introduction to Dynamics, David Griffith's Intrduction to Electrodynamics, Callen's Thermodynamics and R.K. Pathria's Statistical Mechanics. I have done a course in General Relativity in the last semester where I read the Hartle's book on Introduction to Einstein's General Relativity. 

In my third semester I was introduced to Quantum Mechanics. I studied Non Relativistic Quantum Mechanics extensively from David Griffiths Introduction to Quantum Mechanics. To have a better understanding of the mathematics used in the Quantum Mechanics I read the book of Dirac's Principles of Quantum Mechanics. Independently I have also read Lectures on Quantum Theory: Mathematical and Structural Foundations by C.J.Isham and intend to read more from it. At the same time I was lucky to attend a series of lectures on Continuous Groups for physicists given by Prof. Mukunda in IMSc. During this course I learnt about Representation theory of Lie Groups. I read about the various space time groups and their representations namely the Galilean group, Lorentz group and Poincare group. At a later point of time I was able to appreciate those topics much better when I did a course on Relativistic Quantum Mechanics.

To my surprise in the winter that followed that semester, I came to know about Non Hermitian operators. In this regard, I read the article by C.M.Bender and S.Boettcher (Phys. Rev. Lett. 80, 5243 (1998)). I appreciated the importance of N while choosing the potential and how it determined whether the eigen values of the Hamiltonian will be real or complex. To find some application of this kind of a Hamiltonian I read an article on the arXiv (quant-ph/0701141v1) by S. G. Rajeev on Dissipative Mechanics using Complex Valued Hamiltonians.

I have also studied the Adiabatic Theorem in both Classical and Quantum Mechanics in detail. I took the example of a simple harmonic oscillator and found that the ratio of the energy of the system to the frequency of oscillation turns out to be the adiabatic invariant (action variable for a classical system and quantum number (n) for a quantum system). I also read about geometric phases and the Aharonov-Bohm effect. I also read Aharonov and Anandan's paper which removed the condition of adiabaticity from Berry's phase. However Pancharatnam's work on interference of polarized light had gone a step ahead to demonstrate that even a non unitary evolution would lead to the addition of a phase factor.

In the Atomic and Molecular Physics I learnt about the Hartree Fock and Thomas Fermi Statistical model. The way creation and annihilation operators associated with the radiation field (an assembly of quantum harmonic oscillators) are used to calculate the lifetime of spontaneous emission astounded me. During Relativistic Quantum Mechanics course, I read a few chapters of Bjorken and Drell's book on Relativistic Quantum Mechanics and Sakurai's Advanced Quantum Mechanics. I intend to read more on Quantum Field theory and have started reading Schweber's Introduction to Relativistic Quantum Field theory. 

I have read about neutrino oscillations. When neutrinos propagate they change from one lepton flavor to the other which is in violation with the Standard Model. The calculations of the transition probabilities predict that they must have a mass associated with them unlike zero rest mass as predicted by Standard Model. During discussions on electron's interaction with electromagnetic field, I got a glimpse of the famous V-A Theory of Weak Interaction and the way the choice of the interaction was crucial in determining whether parity violations will take place or not. I want to learn more about electroweak interactions. Research to find plausible theories consistent with experimentally obtained data on neutrinos interests me a lot. I am eager to be a part of the search for alternative explanations to Standard Model via the super symmetric theories, theories with extra dimension to remove the shortcomings of the old model. Given my interests, Northwestern is a natural choice for me.

Keeping in mind my long term objective, I would like to pursue a PhD as it would enable me to delve deeper into the subject I am interested in. I believe, Northwestern University with its research facilities and illustrious faculty is the best choice for me. With my diverse academic background and zeal for research, I sincerely hope that I will be given an opportunity to pursue my education and research in your esteemed institution.
\end{document}

